\documentclass[aps,twocolumn,prb,floatfix,amsmath,amssymb,groupedaddress]{revtex4}

\usepackage{graphicx}
\usepackage{psfrag}
\usepackage{epstopdf}
\usepackage{bm}
\usepackage[justification=raggedright]{caption}
\usepackage{subfig}
\usepackage{enumitem}
\usepackage[export]{adjustbox}
\usepackage{mdwlist}
\usepackage{amsthm}
\usepackage{amsmath}
\usepackage{multirow}
\usepackage{breqn}
\newtheorem{prop}{Proposition}
\newtheorem{qsn}{Question}
\makeatletter
\RequirePackage{keyval}
\usepackage{empheq}

\newcommand*\widefbox[1]{\fbox{\hspace{2em}#1\hspace{2em}}}

\begin{document}

\providecommand{\half}{{\frac{1}{2}}}		% I don't use this, but it's a good reminder on how to make a command.

\title{Near Detuned Optical Lattices (NeDOLs), or \\ Snazzy Atomic Near-Detuned Optical Lattices (SANDOLs)}

\author{Hunter Swan}

\maketitle

Definitions:
Saturation intensity: \[I_S = \frac{2\pi^2}{3} \hbar\Gamma \frac{f}{\lambda^2}\]
Saturation parameter: \[s = I/I_S\]

Relevant equations:
Light shift: \[V = \frac{\Omega^2}{4\delta}\]
Scattering rate: \[R = \frac{s\Gamma/2}{1+s+\left(\frac{2\delta}{\Gamma}\right)^2}\]
Rabi frequency: \[\Omega^2 = \half s \Gamma^2\]

Inverting the above gives $s$, $\delta$ in terms of $R$ and $V$:
\[\delta = \frac{\Gamma}{2}\sqrt{\frac{s\Gamma}{2R}-s-1}\]
\[s = V \times \frac{4}{\hbar\Gamma} \sqrt{\frac{s\Gamma}{2R}-s-1}\]


\section{TA's, ASE, and Fabry-Perot filtering}
On resonance, $R = \frac{s\Gamma/2}{1+s}$, so $s=\frac{2R}{\Gamma-2R}\approx \frac{2R}{\Gamma}$, for expected scattering rates $R<<\Gamma$.  For $R=1$ Hz and $\Gamma=2\pi\times 7.4$ kHz, this makes $s=4.3e-5$, so that $I=1.3 e-10$ W/cm$^2$.

If a tapered amplifier (TA) has a given amplified spontaneous emission power spectral density $P$ on resonance, the amount of spontaneous emission power on resonance is of order $P\Gamma$, and if the effective beam size is $A$ the corresponding intensity is $P\Gamma/A$.  Assuming this is $<<I_S$, the resulting scattering rate is $\frac{P\Gamma^2}{2AI_S}$.  To knock this down with a Fabry-Perot (etalon) filter to a preferred level $R$, we supress it by a factor \[\frac{1}{1+F\sin^2\left( 2\pi d cos(\theta) f/c \right)} \rightarrow \frac{1}{1+F\sin^2\left( \pi \delta/FSR \right)},\]
where $F$ is what I call the `fineness' (aka the `coefficient of finesse') and in the former expression $d$ is the mirror separation, $\theta$ the angle of incidence on the etalon, $f$ is the frequency of light, $c$ is the speed of light, and in the latter expression we have specialized to $\theta=0$ and set $f=\delta$ (we assume the etalon is centered to transmit the detuned lattice light, so the frequency distance from resonance to a transmission peak of the etalon is also the detuning $\delta$), and set $c/2d = FSR$, the free spectral range of the etalon. 

Solving for the needed fineness gives \[F = \sin^{-2}\left(\pi\delta/FSR\right) \left[\frac{P\Gamma^2}{2RAI_S}-1\right].\]
Using the relation $\mathcal{F} = \frac{\pi \sqrt{F}}{2}$, where $\mathcal{F}$ is the finesse of the etalon, we have \[\mathcal{F} = \frac{\pi}{2\sin\left(\pi\delta/FSR\right)} \left[\frac{P\Gamma^2}{2RAI_S}-1\right]^{1/2}.\]
Note that this only makes sense if the thing inside the brackets is positive--otherwise you don't need to filter at all.  Also note that the conservative thing to do is to drop the 1 in the brackets anyway. 

Plugging in some numbers, we might expect: 
\begin{itemize}
\item $P=-50$ dBm/MHz, 
\item $R=5$ Hz,
\item $\delta=10$ MHz,
\item FSR$=c/2d=5$ GHz,
\end{itemize}
in which case $\mathcal{F} = 2107$.

\newpage

\section{Quantum mechanics recap}
Here's a collection of useful formulae for later:

Firstly, $[x,f(p)] = i\hbar f'(p)$ and $[p,f(x)] = \frac{\hbar}{i} f'(x)$ for any analytic function $f$ of operators.

We have position and momentum translation operators:
\begin{eqnarray*}
T(a):=\exp(pa/i\hbar) \\
K(b):=\exp(ixb/\hbar)
\end{eqnarray*}
Note that $T(a)^{-1} = T(-a)$ and $K(b)^{-1} = K(-b)$.  Hence $T(a)x T(-a) = x + T(a)[x,T(-a)] = x - T(a)T(-a)a = x+a$, and similarly $K(b)p K(-b) = p - b$.

A note about the interaction picture:  If the Hamiltonian has the form $H = H_0 + K$, where $K$ is some additional term on top of an ``easier'' Hamiltonian $H_0$, then we can eliminate the known time evolution of $H_0$ through the unitary operator \[U=\exp(iH_0t/\hbar).\]
This works because
\begin{eqnarray*}
i\hbar \partial_t U \left|\psi\right> & = & i\hbar (\partial_t U) \left|\psi\right> + i\hbar U \partial_t \left|\psi\right> \\
 & = & - H_0 U \left|\psi\right> + U H \left|\psi\right> \\
& = & \left( U \left(H-H_0\right) U^{-1} \right) U \left|\psi\right>
\end{eqnarray*}
The new operator $U(H-H_0)U^{-1}$ is the effective Hamiltonian in the interaction picture, and $U\left|\psi\right>$ is the wavefunction in the interaction picture.  The term ``interaction picture'' is perhaps overly pretentious--all we have done is essentially change basis. 

\section{2 level atom}
For a two level system with basis $\{|0>,|1>\}$ in an electric field $\vec{E}(t) = \vec{E}_0 \cos{\omega t}$, the Hamiltonian is \[H = \hbar\omega_0 |1><1| - \vec{E}\cdot\vec{d},\] which has the matrix form \[
H = 
\begin{bmatrix}
0 & \hbar\Omega \cos(\omega t) \\
\hbar \Omega^* \cos(\omega t) & \hbar \omega_0 
\end{bmatrix},
\] where $\omega_0$ is the unperturbed energy difference of the two level system, $\vec{d}=-e\vec{r}$ (where $\vec{r}$ is the electron position operator) is the dipole moment operator, and $\Omega:=\frac{1}{\hbar}\vec{E}_0\cdot <1|\vec{d}|0>$ is the `Rabi frequency'.

We move to an interaction picture via the unitary operator \[U = 
\begin{bmatrix}
1 & 0 \\
0 & e^{i\omega_0 t}
\end{bmatrix},
\] which is the time evolution operator $e^{iH_0 t/\hbar}$ for the ``unperturbed'' system with $\vec{E}=0$.  Recall that this transformation works by considering the evolution of $U|\phi>$: \[i\hbar\partial_t U |\phi> = U \left(H - H_0\right) |\phi> = U \left(H - H_0\right)U^{-1} U |\phi>\].

For the Hamiltonian in question, we have
\begin{multline}
U(H-H_0)U^{-1} = \\
\begin{bmatrix}
1 & 0 \\
0 & e^{i\omega_0 t}
\end{bmatrix}
\begin{bmatrix}
0 & \hbar\Omega \cos(\omega t) \\
\hbar \Omega^* \cos(\omega t) & 0 
\end{bmatrix}
\begin{bmatrix}
1 & 0 \\
0 & e^{-i\omega_0 t}
\end{bmatrix}\\
= \begin{bmatrix}
0 & \hbar\Omega \cos(\omega t) e^{-i\omega_0 t} \\
\hbar \Omega^* \cos(\omega t) e^{i\omega_0 t} & 0 
\end{bmatrix} \\
= \begin{bmatrix}
0 & \frac{\hbar\Omega}{2} e^{i\Delta t} \\
\frac{\hbar\Omega^*}{2} e^{-i\Delta t} & 0 
\end{bmatrix} + \begin{bmatrix}
0 & \frac{\hbar\Omega}{2} e^{-i(\omega+\omega_0) t} \\
\frac{\hbar\Omega^*}{2} e^{i(\omega+\omega_0) t} & 0 
\end{bmatrix}\\
=: H' + K,
\end{multline}
where $\Delta:=\omega-\omega_0$ is the ``detuning'', and we have defined the operator with the $e^{i\Delta t}$ as a new effective Hamiltonian $H'$, and called the other matrix $K$.  The latter we will discard in what is called the ``rotating wave approximation'', which is a scale separation effect that works as long as the terms $e^{i(\omega + \omega_0) t}$ oscillate quickly compared to the time scales of $H'$, or explicitly $\omega+\omega_0 >> \Omega$, $\Delta$.  

\subsection{Rabi flopping}

\subsection{The light shift}
If we apply another unitary operator $U = 
\begin{bmatrix}
1 & 0 \\
0 & e^{i\Delta t}
\end{bmatrix}
$ after the one we already applied, the Hamiltonian is transformed to $UHU^{-1} + i\hbar(\partial_t U) U^{-1}$, which for the case of $H'$ above works out to $
\begin{bmatrix}
0 & \frac{\hbar\Omega}{2} \\
\frac{\hbar\Omega^*}{2} & -\hbar\Delta 
\end{bmatrix}.
$  This is time independent!  It has eigenenergies
\begin{equation}
E_{\pm} = -\frac{\hbar\Delta}{2} \pm \frac{\hbar}{2} \sqrt{\Delta^2 + |\Omega|^2}
\end{equation}

For $\Delta >> \Omega$, this becomes 
\begin{eqnarray}
E_{+} & \approx &  \frac{\hbar|\Omega|^2}{4\Delta}\\
E_{-} & \approx &  -\hbar\Delta - \frac{\hbar |\Omega|^2}{4\Delta}
\label{approxLightShift}
\end{eqnarray}

\subsection{A useful relation}
A very useful relation is the following:  Denote $\left<1|\vec{d}|0\right>$ by ${\mu}$.  Then from the so-called ``Wigner-Weiskopf'' theory of spectral emission of atoms, we get that 
\begin{equation}
\hbar\Gamma = \frac{k^3\mu^2}{3\pi\epsilon_0} = \frac{4}{3}k^3\tilde{\mu}^2
\end{equation}
Where the second expression is for SI electrostatic units and the last expression is for rationalized electrostatic units.  ($\tilde{\mu}^2 = \frac{\mu^2}{4\pi\epsilon_0}$, and $k = \frac{2\pi}{\lambda}$.)

A consequence of this is that, when the electric field $\vec{E}$ is parallel to $\mu$, we have 
\begin{eqnarray*}
\hbar^2|\Omega|^2 & = & |\vec{E}|^2\mu^2 = \frac{3\pi\epsilon_0 \hbar\Gamma |\vec{E}|^2}{k^3} \\
& = & \half\hbar^2\Gamma^2 \left(\half c \epsilon_0|\vec{E}|^2\right) \left(\frac{2\pi^2\hbar\Gamma f}{3\lambda^2}\right)^{-1}
\end{eqnarray*}
Identifying $\half c\epsilon_0 |\vec{E}|^2$ as the intensity $I$ for the local electromagnetic field and defining the ``saturation intensity'' as $I_{SAT}:= \frac{2\pi^2}{3}\hbar f \frac{\Gamma}{\lambda^2}$ and the ``saturation parameter'' $s:=I/I_{SAT}$, we have 
\begin{equation}
|\Omega|^2 = \half s \Gamma^2
\label{usefulRelation}
\end{equation}

\section{Optical lattices}
Throughout this section, $E$ will stand for an electric field, $k$ will be a k-vector or k-number, and $w$ will be an angular frequency.  I will drop arrows over vectors, and the juxtaposition of two vectors, like $kx$, will indicate their dot product.

\subsection{Case study}
I'll begin with the special case of two beams with a common, linear polarization and equal frequency and intensity.  All electric field vectors $E$ will be taken real.
\[E = E_0 \cos(k_1x-wt) + E_0 \cos(k_2x-wt)\]
Since the frequencies of the two lasers are equal, $|k_1| = |k_2|=:k$, and I will take the x-axis to be in the direction of $k_1-k_2$.  Name the angle between $k_1$ and $k_2$ to be $\theta$, so that $\cos(\theta) = k_1k_2 / k^2$, and $k_1-k_2 = 2k\sin(\theta/2) \hat{x}$.

Set $K:=k_1-k_2$ and $L:=\half(k_1+k_2)$, so that $k_1 = L+K/2$ and $k_2=L-K/2$.  We have
\begin{eqnarray}
E & = & \frac{E_0}{2} \left(e^{i(k_1x-wt)}+e^{-i(k_1x-wt)}\right)  \nonumber\\
& &  + \frac{E_0}{2} \left(e^{i(k_2x-wt)}+e^{-i(k_2x-wt)}\right)  \nonumber\\
& = & \frac{E_0}{2} \left(e^{ik_1x} + e^{ik_2x}\right) e^{-iwt}  \nonumber\\
& & + \frac{E_0}{2} \left(e^{-ik_1x} + e^{-ik_2x}\right)e^{iwt}  \nonumber\\
& = & \frac{E_0}{2} \left(e^{iKx/2} + e^{-iKx/2}\right) e^{i(Lx-wt)}  \nonumber\\
& & + \frac{E_0}{2} \left(e^{-iKx/2} + e^{iKx/2}\right)e^{-i(Lx-wt)}  \nonumber\\
& = & 2E_0\cos(Kx/2)\cos(Lx-wt)
\label{lattice1d}
\end{eqnarray}

We see that the electric field oscillates sinusoidally at each point in space, with a periodic modulation in field strenth given by $\cos(Kx/2)$.  This spatially periodic modulation of electric field is what we call an optical lattice.  For large detunings, at each point in space the light shifted ground state energy in the presence of the lattice is given by eqs. (\ref{approxLightShift}) and (\ref{usefulRelation}) to be
\[E_- = -\hbar\Delta -\frac{\hbar s \Gamma^2}{8\Delta}\]
Where $s$ is the \textit{local} intensity, proportional to $|E|^2 = |2E_0 \cos(Kx/2)|^2$.  Explicitly, the part of $E_-$ that varies in space (the second term above) has the form
\[\frac{\hbar s_0 \Gamma^2}{2\Delta} \cos^2(Kx/2)\]
Where $s_0 = \half c\epsilon_0 |E_0|^2/I_{SAT}$ is the saturation intensity of either single laser beam. 

Using $\cos^2(a) = \half\left(1+\cos(2a)\right)$ and subtracting off the DC component of $E_-$, we can write the AC part as\begin{equation}
V(x) = \frac{\hbar s_0 \Gamma^2}{4\Delta}\cos(Kx)
\end{equation}
This is a potential for the atom in the lattice.  

Now if we assume that gravity is in the x-direction, the Hamiltonian for the lattice + gravity system is \[H = \frac{p^2}{2m} + mgx + \frac{\hbar s_0 \Gamma^2}{4\Delta}\cos(Kx).\]

We try to eliminate the effect of gravity via an interaction picture, using unitary operator $U = \exp(imgxt/\hbar)$.  The interaction picture Hamiltonian is \[H = \frac{(p-mgt)^2}{2m} + \frac{\hbar s_0 \Gamma^2}{4\Delta}\cos(Kx).\]  We non-dimensionalize this by setting $X:=Kx$, $P:=p / \hbar K$, $E_R:=\hbar^2K^2/2m$, and $T:=E_R t/\hbar = \hbar K^2 t/2m$, giving 
\begin{eqnarray*}
H & = & E_R \left(P-\frac{mgt}{\hbar K}\right)^2 + \frac{\hbar s_0 \Gamma^2}{4 \Delta} \cos(X)\\
& = & E_R \left(P-aT\right)^2 + E_R U \cos(X)
\end{eqnarray*}
Where in the last expression $a = mg/KE_R$ and $U = \hbar s_0 \Gamma^2/4\Delta E_R$.  The time evolution is then
\[i\hbar \partial_t \left|\psi\right> = E_R\left(\left(P-aT\right)^2 + U \cos(X)\right) \left|\psi\right>\]
Or equivalently
\begin{equation}
\boxed{i\partial_T \left|\psi\right> = \left(\left(P-aT\right)^2 + U \cos(X)\right) \left|\psi\right>}
\label{Schrodinger}
\end{equation}
So the non-dimensional Hamiltonian is 
\begin{equation}
\boxed{H = \left(P-aT\right)^2 + U \cos(X)}
\label{Hamiltonian}
\end{equation}
For good measure, I'll box the parameter definitions, too:

\begin{subequations}
\begin{empheq}[box=\widefbox]{align}
E_R &= \frac{\hbar^2 K^2}{2m}  \\
U &= \frac{\hbar s_0 \Gamma^2}{4\Delta E_R} = \frac{s_0 \Gamma^2 m}{2\Delta \hbar K^2} \\
a &=  \frac{mg}{K E_R} = \frac{2m^2 g}{\hbar^2 K^3}  \\
K &=  |k_1-k_2| = 2 k \sin(\theta/2) 
\end{empheq}
\end{subequations}

Writing out these in gory detail gives:
\begin{eqnarray*}
E_R & = &  \frac{8\pi^2\hbar^2 \sin^2(\theta/2)}{m\lambda^2} \\
U & = & \frac{s_0 \Gamma^2 m \lambda^2}{32\pi^2\hbar\Delta\sin^2(\theta/2)} \\
a & = & \frac{m^2 g \lambda^3}{32\pi^3 \hbar^2 \sin^3(\theta/2)} \\
K & = & \frac{4\pi \sin(\theta/2)}{\lambda}
\end{eqnarray*}

Realistic parameters are $\theta\approx 1^{\circ}$, $m=88$ amu, and $\lambda = 689$ nm.  This gives $K\approx 1.6e5$ m$^{-1}$ , $a\approx 9500$, and $E_R/\hbar \approx 9.2$ Hz.  Another useful figure is $a\sin^3(\theta/2) = 6.33e-3$.  With this non-dimensionalized system in hand, we proceed to determine what parameters (i.e. light intensity and detuning) we need to hold atoms against gravity.

\subsubsection{Scattering}
The scattering rate for atoms in the presence of light is 
\begin{equation}
R = \frac{s\Gamma/2}{1+s+\left(\frac{2\Delta}{\Gamma}\right)^2}
\label{scattering}
\end{equation}

Typically, we have a target scattering rate in mind (an upper bound, at least), and so we invert this to find what sort of detuning we should use.  
\begin{equation}
\Delta = \frac{\Gamma}{2} \sqrt{\frac{s\Gamma}{2R} - s - 1} \approx \sqrt{\frac{s\Gamma^3}{8R}}
\label{detuning}
\end{equation}
Where the latter approximation holds as long as $R<<\Gamma$, $s>>1$.  When we want to put a cap on scattering, this expression for $\Delta$ is a lower bound, and thus in the context of a lattice we should interpret $s$ as being the maximum possible intensity anywhere in the lattice, which is given by eq. (\ref{lattice1d}) to be $4s_0$.  Explicitly, \[\Delta = \sqrt{\frac{s_0 \Gamma^3}{2R}}\]
Good values for $R$ are about 10 Hz.

\subsubsection{Tunneling}
If we start an atom in a band zero lattice eigenstate with quasimomentum $q\in[-\half,\half]$, due to the $P-aT$ momentum term in the Hamiltonian, it's quasimomentum is effectively changing in time at rate $-a$.  If it changes slowly enough, the wavefunction will adiabatically track the band zero eigenstate of the lattice.  If it changes too quickly, the wavefunction will tunnel into higher bands.  This tunneling is what we want to avoid.

The bulk of the tunneling occurs at the point of minimal separation between bands zero and one, which occurs at the edge of the first Brillouin zone.  We can estimate the amount of tunneling by something called the Landau-Zener tunneling formula.  It states that the probability amplitude of the tunneled part of the wave function upon crossing the edge of the Brillouin zone is given by \[P = \exp\left(\frac{-\pi \omega_{BG}^2}{4a}\right)\]
Where $\omega_{BG}$ is the frequency of the band gap.  For deep lattices (where the tight binding approximation holds), we can estimate the band gap as follows:  Near the bottom of the potential, the atom sees an approximate harmonic trap.  The harmonic energy level separations are $\sqrt{2U}$, which is thus approximately the band gap.  So we have 
\begin{equation}
P\approx \exp\left(\frac{-\pi U}{2a}\right)
\label{tunneling}
\end{equation}

For a given tunneling probability $P$ and scattering rate $R$, we can solve for the needed detuning $\Delta$ and incident intensity $s_0$.  The relevant equations are 
\begin{eqnarray*}
P & = & \exp\left(\frac{-\pi^2 \hbar s_0 \Gamma^2 \sin(\theta/2)}{4mg\Delta\lambda}\right)\\
\Delta & \approx & \sqrt{\frac{s_0 \Gamma^3}{2R}} \\
\therefore P & \approx & \exp\left(\frac{-\pi^2\hbar\sin(\theta/2)}{4mg\lambda}\sqrt{2Rs_0\Gamma}\right) 
\end{eqnarray*}
So \[s_0 = \frac{-4mg\Delta\lambda\log(P)}{\pi^2\hbar\Gamma^2\sin(\theta/2)} \approx \frac{8m^2g^2\lambda^2\log^2(P)}{\pi^4\hbar^2\sin^2(\theta/2)R\Gamma}\]
Or in in terms of intensity, \[I  = s_0 I_{SAT} \approx \frac{16 m^2g^2 f \log^2(P)}{3\pi^2\hbar\sin^2(\theta/2)R}\]
Plugging in $P=.0001$ and $R=10$ Hz, this works out to $s_0\approx 3.4e6$.  The saturation intensity of the 689 transition in Sr is about 3 $\mu$W/cm$^2$, so the corresponding intensity is $10.2$ W/cm$^2$.

\subsubsection{Blue suppression}
The scattering rate at a given point of space is proportional (roughly) to the light intensity there.  So if an atom is hanging out in a dark spot, it doesn't scatter much.  This can be arranged by using light blue-detuned from a resonance, so that the atoms are pushed towards regions of low intensity.  You want to arrange your relative beam intensities carefully so that the bottoms of the lattice potential have exactly zero intensity, but this can be done.  We can estimate the effective scattering rate in this scenerio as follows:

The spatially varying dimensionless intensity in a lattice is given by $4 s_0 \cos^2(Kx/2)$, and near the bottom of a well this is approximately given by $s_0K^2x^2$.  The Hamiltonian is $\frac{p^2}{2m} + mgx + \frac{\hbar s_0 \Gamma^2}{4\Delta}\cos(Kx)$, which near a well bottom is approximately 
\begin{eqnarray*}
H & \approx & \frac{p^2}{2m} + mgx + \frac{\hbar s_0 \Gamma^2}{4\Delta} \times\half K^2x^2 \\
& = & \frac{p^2}{2m} + \frac{\hbar s_0\Gamma^2 K^2}{8\Delta} \left(x + \frac{4mg\Delta}{\hbar s_0\Gamma^2K^2}\right)^2 - \frac{2m^2g^2\Delta}{\hbar s_0 \Gamma^2 K^2} \\
& =: & \frac{p^2}{2m} + \half m\omega^2 (x+x_0)^2 + C
\end{eqnarray*}
Where in the final expression we have defined $\omega := \sqrt{\frac{\hbar s_0 \Gamma^2 K^2}{4m\Delta}}$ and $x_0 := \frac{4mg\Delta}{\hbar s_0 \Gamma^2 K^2} = \frac{g}{\omega^2}$ ($C$ being some constant offset that will be dropped from here onwards).  This is a harmonic oscillator potential, which has ground state wavefunction
\begin{equation*}
\psi(x) = \left(\frac{m\omega}{\pi\hbar}\right)^{1/4}\exp\left(- \frac{m\omega}{2\hbar}\left(x+x_0\right)^2\right)
\end{equation*}
For such a wavefunction, the average dimensionless intensity seen by the atom is given by 
\[\int \psi(x)^2 s_0 K^2 x^2 dx = s_0 K^2 \left(\frac{\hbar}{2m\omega} + x_0^2\right)\]
Plugging in the expressions for $\omega$ and $x_0$ this becomes
\[ s_0 \left( \left(\frac{4mg\Delta}{\hbar s_0\Gamma^2 K}\right)^2 + \sqrt{\frac{\hbar\Delta K^2}{ms_0\Gamma^2}}\right) = s_0 \left( \left(\frac{a}{U}\right)^2 + \sqrt{\frac{1}{2U}}\right)\]

For the present case, $\left(\frac{a}{U}\right)^2 >> \sqrt{\frac{1}{2U}}$, so the effective intensity is $\approx s_0 \left(\frac{a}{U}\right)^2$, which in turn equals (via eq. (\ref{tunneling})) $\frac{-\pi s_0}{2\log(P)}$.  Inserting this into eq. (\ref{detuning}) for the needed detuning to achieve a given scattering rate, we find
\[\Delta\approx \sqrt{\frac{-\pi s_0\Gamma^3}{16R\log(P)}}\]
Thus 
\[P\approx\exp\left(\frac{-\pi^{3/2}\hbar\sin(\theta/2)}{mg\lambda}\sqrt{-Rs_0\Gamma \log(P)}\right) \]
Solving for $s_0$, we get
\begin{equation}
s_0 = \frac{-\log(P)}{R\Gamma} \left(\frac{mg\lambda}{\pi^{3/2}\hbar \sin(\theta/2)}\right)^2
\end{equation}

For $R=10$ Hz, $\theta=1^{\circ}$, and $P=10^{-4}$ on the Sr 689 line, this works out to $3.3 \times 10^5$, a factor of 10 better than the red detuned case.

\subsubsection{Kinematics}
We want to use the lattice to transport atoms.  For two static beams with width $w$ interesecting at angle $\theta$, the region in which they overlap has length $L = w/\sin(\theta/2)$.  Thus the power required to transport atoms over distance $L$ can be estimated as \[w^2 s_0 I_{SAT} = L^2 \sin^2(\theta/2) s_0 I_{SAT} = \frac{16 L^2 m^2g^2 f \log^2(P)}{3\pi^2\hbar R}\]
For $L=1$ m, $R=10$ Hz, and $P=10^{-4}$ this works out to 7.6 Watts.

An alternative approach is to put one beam pointed horizontal and the other pointed just above horizontal by angle $\theta/2$, and to accelerate the lattice so as to cancel gravity.  For two beams of width $w$, the region where they overlap is a parallelogram of width $L = \half w /\sin(\theta/2)$.  The lattice itself is inclined at angle $\theta/4$ relative to gravity, so the total acceleration is given by $A+g\sin(\theta/4)$.  We want the net downwards acceleration of the atoms, $A\sin(\theta/4)-g$, to be zero, so total acceleration is $g\sin(\theta/4) + g/\sin(\theta/4)$.  For small angles $\theta$, this is $\approx g/\sin(\theta/4)$, and the total power required is 
\[w^2 s_0 I_{SAT} \approx \frac{256 L^2 m^2g^2 f \log^2(P)}{3\pi^2\hbar R}\]
which is 16 times larger than the above method.

Another approach is to launch the atoms and catch them on the other side.  The amount of power to launch them distance $L$ works out to \[\approx \frac{4 L^2 m^2g^2 f \log^2(P)}{3\pi^2\hbar R \cos(\theta/4)},\] which is a factor of 4 better than the first approach.


\subsubsection{Moving States}
In the presense of non-zero acceleration $a$, the instantaneous eigenstates of the Hamiltonian, which we denote by $\left|n,q\right>$ for band $n$ at quasimomentum $q$, are not the most ``stationary'' states available--if you start a wavefunction in such a state, it will oscillate back and forth between other states, more so than it needs to.  To figure out what are the ``most stationary" states of the moving system, expand a wavefunction as
\[\left|\psi\right> = \sum_m c_m(t) \left|m,q(t)\right>,\]
where $q(t) = q_0 - at$ is the instantaneous quasimomentum.  Applying the Schrodinger equation to this gives
\begin{eqnarray*}
i\partial_t\left|\psi\right> & = & i \sum_m \dot{c}_m(t) \left|m,q(t)\right> + c_m(t)\partial_t \left|m,q(t)\right> \\
& = & H_{q(t)} \sum_m c_m(t) \left|m,q(t)\right> \\
& = & \sum_m E_{m,q(t)} c_m(t) \left|m,q(t)\right>
\end{eqnarray*}
Therefore
\begin{eqnarray*}
i\dot{c}_n & = & \sum_m \left(E_{m,q} \delta_{nm} - i \left<n,q\right| \partial_t \left|m,q\right>\right) c_m \\
& = & \sum_m \left(E_{m,q} \delta_{nm} - i \frac{dq}{dt} \left<n,q\right| \partial_q \left|m,q\right>\right) c_m \\
\end{eqnarray*}

In our case, $\frac{dq}{dt} = -a$.  To evaluate this, we need to figure out how to deal with the $\left<n,q\right|\partial_q\left|m,q\right>$ factor.  To do so, begin with the relation $H_q \left|m,q\right> = E_{m,q} \left|m,Q\right>$ and expand $q =: Q + dq$ around $Q$ to get 
\begin{eqnarray*}
H_{Q+dq} \left|m,Q+dq\right> & = & \left(H_Q + dq \partial_qH_Q + \cdots\right) \\
& & \times\left(\left|m,Q\right>+dq\partial_q\left|m,Q\right> + \cdots\right) \\
& = & H_Q\left|m,Q\right> \\
& + & dq\left(H_Q\partial_q\left|m,Q\right> + \left(\partial_qH_Q\right)\left|m,Q\right>\right) \\
& + & \mathcal{O}(dq^2) \\
\end{eqnarray*}
and
\begin{eqnarray*}
\!\!\!E_{m,Q+dq} \left|m,Q+dq\right> & = & \left(E_{m,Q} + dq \partial_q E_{m,Q} + \cdots\right) \\
& & \times \left(\left|m,Q\right> + dq \partial_q \left|m,Q\right>\right) \\
& = & E_{m,Q} \left|m,Q\right> \\
& + & dq \left(E_{m,Q}\partial_q\left|m,Q\right> + \left|m,Q\right>\partial_q E_{m,Q}\right) \\
& + & \mathcal{O}(dq^2)
\end{eqnarray*}

The zeroth order tems above match up automatically.  Equating the first order terms gives
\[\!\!\!\!\!H_Q\partial_q\left|m,Q\right> + \left(\partial_qH_Q\right)\left|m,Q\right> = E_{m,Q}\partial_q\left|m,Q\right> + \left|m,Q\right>\partial_q E_{m,Q}\]
so that
\begin{eqnarray*}
& & \left<n,Q\right|\partial_q\left|m,Q\right> E_{m,Q} + \partial_q E_{m,Q} \delta_{nm} \\
& = & \left<n,Q\right|H_Q\partial_q\left|m,Q\right> + \left<n,Q\right|\left(\partial_q H_Q\right)\left|m,Q\right> \\
& = & E_{Q,n}\left<n,Q\right|\partial_q\left|m,Q\right> + \left<n,Q\right|\left(\partial_q H_Q\right)\left|m,Q\right> \\
\end{eqnarray*}
therefore
\begin{eqnarray*}
& \left(E_{n,Q}-E_{m,Q}\right) \left<n,Q\right|\partial_q\left|m,Q\right> \\
 & = \partial_q E_{m,Q}\delta_{nm}-\left<n,Q\right|\left(\partial_q H_Q\right)\left|m,Q\right>
\end{eqnarray*}
For $n=m$, this says that $\partial_q E_{n,Q} = \left<n,Q\right|\left(\partial_q H_Q\right)\left|n,Q\right>$.  For $n\neq m$, assuming a non-degenerate spectrum, we have 
\begin{equation}
\left<n,Q\right|\partial_q\left|m,Q\right> = \frac{\left<n,Q\right|\left(\partial_q H_Q\right)\left|m,Q\right>}{E_{m,Q}-E_{n,Q}}
\end{equation}

Interestingly, the diagonal element $\left<n,Q\right|\partial_q\left|n,Q\right>$ is generally not well-defined.  This is because the eigenstate $\left|n,q\right>$ is only defined up to a phase.  If we replace it by $e^{i\theta(q)}\left|n,q\right>$, where $\theta(q)$ is a smooth, real function, then the corresponding diagonal element becomes $\left<n,Q\right|e^{-i\theta(q)}\partial_qe^{i\theta(q)}\left|n,Q\right> = \left<n,Q\right|\partial_q\left|n,Q\right> + i\theta'(q)$, which can thus be set to any desired imaginary value by appropriate choice of $\theta(q)$.  The real part, however, must be zero because 
\begin{eqnarray*}
0 & = & \partial_q 1 \\
& = & \partial_q \left<n,Q\right|\left|n,Q\right> \\
& = & \left( \partial_q \left<n,Q\right|\left|n,Q\right>\right) + \left( \left<n,Q\right|\partial_q\left|n,Q\right>\right) \\
& = & \left( \left<n,Q\right|\partial_q\left|n,Q\right>\right)^* \left( \left<n,Q\right|\partial_q\left|n,Q\right>\right) \\
& = & 2 \Re\left( \left<n,Q\right|\partial_q\left|n,Q\right>\right)
\end{eqnarray*}

Now in the present lattice case, the Hamiltonian can be viewed as $H_q = (q+P_0)^2 + U\cos(X)$, where $P_0$ is the momentum operator for quasimomentum $0$.  So $\partial_q H_Q = 2(q+P_0)$.  Thus the time evolution for $c_n$ is given by
\begin{equation}
i\dot{c}_n = \mathcal{H}_{q(t),nm}c_m
\end{equation}
where the Hermitian ``moving Hamiltonian" is given by
\begin{equation}
\mathcal{H}_{q(t),nm} = -i\frac{dq}{dt} \frac{\left<n,q(t)\right|2\left(q(t)+P_0\right)\left|m,q(t)\right>}{E_{m,q(t)}-E_{n,q(t)}}
\end{equation}
for $n\neq m$.  We set the diagonal elements of $\mathcal{H}$ to zero (or any other convenient value) as described above.  An eigenstate of this matrix at a given time $t$ looks ``instantaneously stationary".  This is because if $\mathcal{H}c = \alpha c$, then $\dot{c} = -i\alpha c$, which means the amplitudes $|c_n|^2$ are not changing in time.

\subsection{General case}
The electric field of a bunch of plane waves is
\begin{eqnarray*}
\vec{E}(x,t) & = & \sum_j \Re\{\vec{E}_j e^{i(k_j x - w_j t)}\}
\end{eqnarray*}

\end{document}
